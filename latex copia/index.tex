El nostre objectiu de la pràctica, consistirà en\+: Construir un programa modular, el qual ens permeti construir un arbre filogènic per un conjunt N d\textquotesingle{}espècies uttilitzant un mètode aproximatiu anomenat W\+P\+G\+MA.

I per introduir-\/nos en el tema, els arbres filogènics, són diagrames i ens mostraran les hipotètiques relacions evolutives d\textquotesingle{}un conjunt d\textquotesingle{}espècies. ~\newline
 També cal recalcar que en cap cas seran fets definitius, provats, sinó hipòtesis de com possiblement van evolucionar a partir d\textquotesingle{}un avantapassat en comú. ~\newline
 Per poder generar un arbre, sempre necessitarem la informació obtinguda a partir de comparar i analitzar moltes característiques de les espècies, com també les proteïnes i les seqüències d\textquotesingle{}A\+DN. ~\newline
 Ja que si podem seqüenciar l\textquotesingle{}A\+DN, ens permetrà, augmentar l\textquotesingle{}habilitat de comparar gens entre espècies.

~\newline
 Trobarem les espècies a les fulles de l\textquotesingle{}arbre, i formaran juntes les arrels de l\textquotesingle{}arbre i per identificar un avantpassat en comú, en aquesta pràctica, per nosaltres serà un punt de ramificació, un node, una intersecció on divergiran dos grups de descendència. ~\newline
 Introduirem les classes\+: {\itshape \mbox{\hyperlink{class_especie}{Especie}} }, {\itshape  \mbox{\hyperlink{class_cjt___especies}{Cjt\+\_\+\+Especies}} }, {\itshape  \mbox{\hyperlink{class_cluster}{Cluster}} } i {\itshape  \mbox{\hyperlink{class_cjt___clusters}{Cjt\+\_\+\+Clusters}} } 